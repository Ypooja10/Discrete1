
\documentclass{article}
\usepackage{graphicx} % Required for inserting images
\usepackage{amsmath}
\usepackage{tfrupee}


\title{discrete}
\author{Pooja yadav}
\date{September 2023}

\begin{document}

\maketitle
\section{30-c-3}

\begin{enumerate}

\item If  \(\frac{-5}{7}\)
 , a, 2 are consecutive terms in an arithmetic progression then the value of 'a' is:
   \begin{enumerate}
   
    \item \(\frac{9}{7}\)
    \item \(\frac{9}{14}\)
    \item \(\frac{19}{7}\)
    \item \(\frac{19}{14}\)


\end{enumerate}

 \item If two positive integers p and q can be expressed as \[p = ab^3\] and \[q = a^2b\] a and b being prime numbers, then find LCM of (p,q).

\item Show that any positive odd integer is of the form 4q + 1 or 4q + 3 for some integer q.
\item proves that $\sqrt{5}$ is an irrational number.


\end{enumerate}
\section{30-1-3}
\begin{enumerate}
   
   
    \item[1(a)] Find the sum of the first twelve 2-digit numbers which are multiples of 6.  
    \begin{center}
    OR
\end{center}
    \item[(b)] In an AP, if $a_2 = 26$ and $a_{15} = -26$ then write the AP.
 \begin{center}
 
    case study
\end{center}
\item[2.] In mathematics, relations can be expressed in various ways. The matchstick patterns are based on linear relations. different strategies can be used to calculate the number of matchsticks used in different figures.\\
 
  One such pattern is shown below. Observe the pattern and answer the following questions using arithmetic progression.

  \includegraphics[width=12cm, height=4cm]{ /storage/emulated/0/Android/media/com.whatsapp/WhatsApp/Media/WhatsApp Images/IMG-20230924-WA0003.jpg  }
   

\begin{enumerate}

\item[(a)]Write the AP for the number of triangles used in the figures. Also
write the $n^{th}$ term of this AP.
\item[(b)] Which figure has 61 matchsticks.?
\end{enumerate}
\end{enumerate}


\section{30-2-3}
\begin{enumerate}
    \item[1(a)] In an A.P. if the sum of the third and seventh term is zero, finds its $5^{th}$
    term.
    \begin{center}
    OR
\end{center}
\item[(b)] Determine the A.P. whose third term is 5 and the seventh term is 9.
\item [2.]Find the sum of the first 20 terms of an A.P. whose nth term is given as \[a_n= 5 - 2n.\]
\end{enumerate}
\section{30-3-3}
\begin{enumerate}
    \item Find the common difference 'd' of an AP whose first term is 10 and the sum of the first 14 terms is 1505.
    \item For what value of 'n', are the $n^{th}$ terms of the AP: 9,7,5...and 15,12,9...the same?
\end{enumerate}

\section{30-4-3}
\begin{enumerate}
    \item Find the sum of all 11 terms of A.P. whose $6^{th}$ term is 30.
    \item The sum of the first three terms of A.P. is 33. If the product of the first and third term exceeds the second term by 29, find the A.P.
\end{enumerate}
\section{30-B-5}
\begin{enumerate}
    \item[1(a)]Find the number of terms in the following AP:
    \begin{center}
    5,11,17,............,203
\end{center}
\begin{center}
    OR
\end{center}
    \item[(b)] Find the sum of the first 20 terms of an AP whose $n^{th}$ is given as \[a_n= 5 - 3n\]
   \begin{center}

   case study
\end{center}
    \item[2.] While buying an expensive item like a house or a car, it becomes easier for a middle-class person to take a loan from a bank and then repay the loan along with interest in easy installments.\hfill \break
   
Aman buys a car by taking a loan of \rupee 2,36,000 from the bank and starts repaying the loan in monthly installments. He pays \rupee 2000 as the first installment and then increases the installments by \rupee 500 every month.    

 \begin{enumerate}
   
    \item[(a)]Find the amount he pays in the $25^{th}$ installments.
    \item [(b)]Find the total amount paid by him in the first 25 installments.



\end{enumerate}
\end{enumerate}
\end{document}
